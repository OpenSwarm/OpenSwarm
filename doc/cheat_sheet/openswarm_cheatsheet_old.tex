\documentclass[10pt,a4paper]{article}

% Packages
\usepackage{fancyhdr}           % For header and footer
\usepackage{multicol}           % Allows multicols in tables
\usepackage{tabularx}           % Intelligent column widths
\usepackage{tabulary}           % Used in header and footer
\usepackage{hhline}             % Border under tables
\usepackage{graphicx}           % For images
\usepackage{xcolor}             % For hex colours
%\usepackage[utf8x]{inputenc}    % For unicode character support
\usepackage[T1]{fontenc}        % Without this we get weird character replacements
\usepackage{colortbl}           % For coloured tables
\usepackage{setspace}           % For line height
\usepackage{lastpage}           % Needed for total page number
\usepackage{seqsplit}           % Splits long words.
%\usepackage{opensans}          % Can't make this work so far. Shame. Would be lovely.
\usepackage[normalem]{ulem}     % For underlining links
% Most of the following are not required for the majority
% of cheat sheets but are needed for some symbol support.
\usepackage{amsmath}            % Symbols
\usepackage{MnSymbol}           % Symbols
\usepackage{wasysym}            % Symbols
\usepackage[english,german]{babel}              % Languages
\usepackage[draft]{hyperref}
\usepackage{xspace}%to have an optional space after a \OS


\newcommand{\OS}{\href{http://www.OpenSwarm.org}{OpenSwarm}\xspace}
\newcommand{\me}{\href{http://trenkwalder.tech}{Stefan M. Trenkwalder}\xspace}
\newcommand{\version}{v0.15.9.15\xspace}
\newcommand{\license}{\href{http://openswarm.org/license/}{adapted FreeBSD License}\xspace}

% Document Info
\author{Stefan M. Trenkwalder}
\pdfinfo{
  /Title (OpenSwarm - Cheat Sheet)
  /Creator (OpenSwarm)
  /Author (Stefan M. Trenkwalder - http://trenkwalder.tech)
  /Subject (OpenSwarm Cheat Sheet)
}

% Lengths and widths
\addtolength{\textwidth}{6cm}
\addtolength{\textheight}{-1cm}
\addtolength{\hoffset}{-3cm}
\addtolength{\voffset}{-2cm}
\setlength{\tabcolsep}{0.2cm} % Space between columns
\setlength{\headsep}{-12pt} % Reduce space between header and content
\setlength{\headheight}{85pt} % If less, LaTeX automatically increases it
\renewcommand{\footrulewidth}{0pt} % Remove footer line
\renewcommand{\headrulewidth}{0pt} % Remove header line
\renewcommand{\seqinsert}{\ifmmode\allowbreak\else\-\fi} % Hyphens in seqsplit
% This two commands together give roughly
% the right line height in the tables
\renewcommand{\arraystretch}{1.3}
\onehalfspacing

% Commands
\newcommand{\SetRowColor}[1]{\noalign{\gdef\RowColorName{#1}}\rowcolor{\RowColorName}} % Shortcut for row colour
\newcommand{\mymulticolumn}[3]{\multicolumn{#1}{>{\columncolor{\RowColorName}}#2}{#3}} % For coloured multi-cols
\newcolumntype{x}[1]{>{\raggedright}p{#1}} % New column types for ragged-right paragraph columns
\newcommand{\tn}{\tabularnewline} % Required as custom column type in use

% Font and Colours
\definecolor{HeadBackground}{HTML}{333333}
\definecolor{FootBackground}{HTML}{666666}
\definecolor{TextColor}{HTML}{333333}
\definecolor{DarkBackground}{HTML}{0348A3}
\definecolor{LightBackground}{HTML}{D4DFEF}
\renewcommand{\familydefault}{\sfdefault}
\color{TextColor}

% Header and Footer
\pagestyle{fancy}
\fancyhead{} % Set header to blank
\fancyfoot{} % Set footer to blank

\fancyhead[L]{
\noindent
\begin{multicols}{2}
\begin{tabulary}{5.8cm}{C}
    \SetRowColor{DarkBackground}
    \vspace{-7pt}
    {\parbox{\dimexpr\textwidth-2\fboxsep\relax}{\noindent
       \hspace*{-6pt}\includegraphics[width=2cm]{../img/openswarm_logo.png}
    }
\end{tabulary}
\columnbreak
\begin{tabulary}{11cm}{L}
    \vspace{-2pt}\large{\bf{\textcolor{DarkBackground}{\textrm{\OS Cheat Sheet}}}} \\
    \normalsize{by \textcolor{DarkBackground}{\me} via \textcolor{DarkBackground}{\uline{\url{http://openswarm.org}}}}
\end{tabulary}
\end{multicols}}

\fancyfoot[L]{ \footnotesize
\noindent
\begin{multicols}{3}
\begin{tabulary}{5.8cm}{LL}
  \SetRowColor{FootBackground}
  \mymulticolumn{2}{p{5.377cm}}{\bf\textcolor{white}{\OS}} \\
  {An event-driven and pre-emptive operating system for miniature robots}  \\
  \vspace{-2pt}\url{http://openswarm.org}\\
  \end{tabulary}
\vfill
\columnbreak
\begin{tabulary}{5.8cm}{L}
  \SetRowColor{FootBackground}
  \mymulticolumn{1}{p{5.377cm}}{\bf\textcolor{white}{Cheat Sheet}}  \\
   \vspace{-2pt} 19th February, 2016.\\
   Page {\thepage} of \pageref{LastPage}.
\end{tabulary}
\vfill
\columnbreak
\begin{tabulary}{5.8cm}{L}
  \SetRowColor{FootBackground}
  \mymulticolumn{1}{p{5.377cm}}{\bf\textcolor{white}{Author}}  \\
  \SetRowColor{white}
  \vspace{-5pt}
  %\includegraphics[width=48px,height=48px]{dave.jpeg}
  \me\\
  \url{http://trenkwalder.tech}
\end{tabulary}
\end{multicols}}




\begin{document}
\raggedright
\raggedcolumns

% Set font size to small. Switch to any value
% from this page to resize cheat sheet text:
% www.emerson.emory.edu/services/latex/latex_169.html
\footnotesize % Small font.

\begin{multicols*}{2}

\begin{tabularx}{8.4cm}{X}
\SetRowColor{DarkBackground}
\mymulticolumn{1}{x{8.4cm}}{\bf\textcolor{white}{Legende}}  \tn
% Row 0
\SetRowColor{LightBackground}
\mymulticolumn{1}{x{8.4cm}}{{\bf{STRG + x \textgreater{} y}} Bedeutet das die STRG-Taste und die Taste für den jeweilig ersten Buchstaben drücken, gefolgt vom zweiten Buchstaben nach der spitzen Klammer} \tn 
% Row Count 4 (+ 4)
% Row 1
\SetRowColor{white}
\mymulticolumn{1}{x{8.4cm}}{{\bf{Fettgeschriebenes}} sind T{\"a}tigkeiten, welche ausgeführt werden müssen bevor der Shortcut arbeitet.} \tn 
% Row Count 7 (+ 3)
% Row 2
\SetRowColor{LightBackground}
\mymulticolumn{1}{x{8.4cm}}{{\emph{Schr{\"a}ggestellter Text}} bezeichnet Tastatureingaben, die vom User abh{\"a}ngig sind.} \tn 
% Row Count 9 (+ 2)
\hhline{>{\arrayrulecolor{DarkBackground}}-}
\end{tabularx}
\par\addvspace{1.3em}

\begin{tabularx}{8.4cm}{X}
\SetRowColor{DarkBackground}
\mymulticolumn{1}{x{8.4cm}}{\bf\textcolor{white}{Bemerkungen}}  \tn
\SetRowColor{white}
\mymulticolumn{1}{x{8.4cm}}{Wenn hinter einem Shortcut erw{\"a}hnt wird, dass dieser nur auf englischen Tastaturen funktioniert, so liegt dies daran, das Sublime die Anschl{\"a}ge direkt von der Tastatur nimmt und nicht vom System. Wird ein Shortcut für das eigene Tastaturlayout ben{\"o}tigt muss man sich diese Shortcuts selbst definieren. Wie dies geht, steht in der (engl.) Dokumentation. Auf der deutschen Sublime Seite \{\{link="http://sublime.pixel-anarchy.de"\}\}sublime.pixel-anarchy.de\{\{/link\}\} ist dieser Punkt noch nicht vorhanden, wird aber bald als Tutorial dort zu finden sein.% Row Count 12 (+ 12)
} \tn 
\hhline{>{\arrayrulecolor{DarkBackground}}-}
\end{tabularx}
\par\addvspace{1.3em}

\begin{tabularx}{8.4cm}{x{4.48 cm} x{3.52 cm} }
\SetRowColor{DarkBackground}
\mymulticolumn{2}{x{8.4cm}}{\bf\textcolor{white}{Grundfunktionen}}  \tn
% Row 0
\SetRowColor{LightBackground}
Akt. Auswahl kopieren & STRG + c \tn 
% Row Count 1 (+ 1)
% Row 1
\SetRowColor{white}
Akt. Zwischenablage einfügen & STRG + v \tn 
% Row Count 3 (+ 2)
% Row 2
\SetRowColor{LightBackground}
Zwischenablage mit richtigem Einzug einfügen & STRG + SHIFT + v \tn 
% Row Count 6 (+ 3)
% Row 3
\SetRowColor{white}
Undo/Redo & STRG + Z / STRG + B \tn 
% Row Count 8 (+ 2)
% Row 4
\SetRowColor{LightBackground}
Auswahl "Autocomplete Vorschl{\"a}ge" best{\"a}tigen & STRG + SPACE \tn 
% Row Count 11 (+ 3)
% Row 5
\SetRowColor{white}
Akt. Zeile einrücken & STRG + {]} (nur engl. Tastaturen) \tn 
% Row Count 13 (+ 2)
% Row 6
\SetRowColor{LightBackground}
Einrückung der akt. Zeile entfernen & STRG + {[} (nur engl. Tastaturen) \tn 
% Row Count 15 (+ 2)
% Row 7
\SetRowColor{white}
Akt. Zeile mit übergeordneter vertauschen & STRG + \{\{fa-arrow-up\}\} \tn 
% Row Count 17 (+ 2)
% Row 8
\SetRowColor{LightBackground}
Akt. Zeile mit daruntereliegender vertauschen & STRG + \{\{fa-arrow-down\}\} \tn 
% Row Count 20 (+ 3)
% Row 9
\SetRowColor{white}
N{\"a}chsten Buchstaben l{\"o}schen & entf \tn 
% Row Count 22 (+ 2)
% Row 10
\SetRowColor{LightBackground}
vorhergehenden Buchstaben l{\"o}schen & Backspace \tn 
% Row Count 24 (+ 2)
% Row 11
\SetRowColor{white}
N{\"a}chstes Wort l{\"o}schen & STRG + entf \tn 
% Row Count 26 (+ 2)
% Row 12
\SetRowColor{LightBackground}
Vorhergehendes Wort l{\"o}schen & STRG + Backspace \tn 
% Row Count 28 (+ 2)
\hhline{>{\arrayrulecolor{DarkBackground}}--}
\end{tabularx}
\par\addvspace{1.3em}

\begin{tabularx}{8.4cm}{x{3.68 cm} x{4.32 cm} }
\SetRowColor{DarkBackground}
\mymulticolumn{2}{x{8.4cm}}{\bf\textcolor{white}{GUI und Layout}}  \tn
% Row 0
\SetRowColor{LightBackground}
Schriftgr{\"o}{\ss}e erh{\"o}hen & STRG + "+" \tn 
% Row Count 2 (+ 2)
% Row 1
\SetRowColor{white}
Schriftgr{\"o}{\ss}e verringern & STRG + "-" \tn 
% Row Count 4 (+ 2)
% Row 2
\SetRowColor{LightBackground}
Zu Panel wechseln & STRG + {\bf{Panel Nummer}} \tn 
% Row Count 6 (+ 2)
% Row 3
\SetRowColor{white}
Datei auf anderes Panel setzen & STRG + SHIFT + {\bf{Panel Nummer}} \tn 
% Row Count 8 (+ 2)
% Row 4
\SetRowColor{LightBackground}
Vollbild & F11 \tn 
% Row Count 9 (+ 1)
% Row 5
\SetRowColor{white}
Ablenkungsfreier Modus & SHIFT + F11 \tn 
% Row Count 11 (+ 2)
% Row 6
\SetRowColor{LightBackground}
Konsole \seqsplit{anzeigen/verbergen} & STRG + ` (nur auf engl. Tastaturen) \tn 
% Row Count 13 (+ 2)
% Row 7
\SetRowColor{white}
Sidebar \seqsplit{anzeigen/verbrergen} & STRG + k \textgreater{} b \tn 
% Row Count 15 (+ 2)
% Row 8
\SetRowColor{LightBackground}
1/2/3/4-Spalten Layout & ALT + SHIFT + 1 ... 4 \tn 
% Row Count 17 (+ 2)
% Row 9
\SetRowColor{white}
2/3-Zeilen Layout & ALT + SHIFT + 8/9 \tn 
% Row Count 18 (+ 1)
% Row 10
\SetRowColor{LightBackground}
x2 Grid Layout & ALT + SHIFT + 5 \tn 
% Row Count 19 (+ 1)
% Row 11
\SetRowColor{white}
Neues Fenster & STRG + SHIFT + n \tn 
% Row Count 20 (+ 1)
\hhline{>{\arrayrulecolor{DarkBackground}}--}
\end{tabularx}
\par\addvspace{1.3em}

\begin{tabularx}{8.4cm}{x{2.56 cm} x{5.44 cm} }
\SetRowColor{DarkBackground}
\mymulticolumn{2}{x{8.4cm}}{\bf\textcolor{white}{Code einklappen}}  \tn
% Row 0
\SetRowColor{LightBackground}
Sektion einklappen & STRG + SHIFT+ {[} (nur engl. Tastaturen) \tn 
% Row Count 2 (+ 2)
% Row 1
\SetRowColor{white}
Sektion ausklappen & STRG + SHIFT + {]} (nur engl. Tastaturen) \tn 
% Row Count 4 (+ 2)
% Row 2
\SetRowColor{LightBackground}
Alles einklappen & STRG + k \textgreater{} 1 \tn 
% Row Count 6 (+ 2)
% Row 3
\SetRowColor{white}
Alles ausklappen & STRG + k \textgreater{} j \tn 
% Row Count 8 (+ 2)
% Row 4
\SetRowColor{LightBackground}
\seqsplit{Einklapplevel} & STRG + 2 ... 9 \tn 
% Row Count 10 (+ 2)
\hhline{>{\arrayrulecolor{DarkBackground}}--}
\end{tabularx}
\par\addvspace{1.3em}

\begin{tabularx}{8.4cm}{x{3.92 cm} x{4.08 cm} }
\SetRowColor{DarkBackground}
\mymulticolumn{2}{x{8.4cm}}{\bf\textcolor{white}{Navigation}}  \tn
% Row 0
\SetRowColor{LightBackground}
Neue Datei & STRG + n \tn 
% Row Count 1 (+ 1)
% Row 1
\SetRowColor{white}
Datei {\"o}ffnen & STRG + o \tn 
% Row Count 2 (+ 1)
% Row 2
\SetRowColor{LightBackground}
Aktuellen Reiter schliessen & STRG + w \tn 
% Row Count 4 (+ 2)
% Row 3
\SetRowColor{white}
Öffne "Goto Anything" Panel & STRG + p \tn 
% Row Count 6 (+ 2)
% Row 4
\SetRowColor{LightBackground}
Reiter wechseln & STRG +  {\emph{Reiternummer}} \tn 
% Row Count 8 (+ 2)
% Row 5
\SetRowColor{white}
Gehe zu Zeile in aktuellem Reiter & STRG + G \tn 
% Row Count 10 (+ 2)
% Row 6
\SetRowColor{LightBackground}
Gehe zu Zeile in anderem Dokument & STRG + p \textgreater{} {\emph{Dateiname}} {\bf{:}} {\emph{Zeile}} \tn 
% Row Count 12 (+ 2)
% Row 7
\SetRowColor{white}
Zum n{\"a}chsten Reiter wechseln & STRG + Bild \{\{fa-arrow-down\}\} \tn 
% Row Count 14 (+ 2)
% Row 8
\SetRowColor{LightBackground}
Zum vorhergeehenden Reiter wechseln & STRG + Bild \{\{fa-arrow-up\}\} \tn 
% Row Count 16 (+ 2)
% Row 9
\SetRowColor{white}
Springen zwischen den Reitern & STRG + Tabulator \tn 
% Row Count 18 (+ 2)
\hhline{>{\arrayrulecolor{DarkBackground}}--}
\end{tabularx}
\par\addvspace{1.3em}

\begin{tabularx}{8.4cm}{x{4 cm} x{4 cm} }
\SetRowColor{DarkBackground}
\mymulticolumn{2}{x{8.4cm}}{\bf\textcolor{white}{Auswahlen}}  \tn
% Row 0
\SetRowColor{LightBackground}
Multiauswahlen & STRG + {\bf{Linksklick}} \tn 
% Row Count 2 (+ 2)
% Row 1
\SetRowColor{white}
W{\"a}hl die akt. Zeile aus & STRG + l \tn 
% Row Count 4 (+ 2)
% Row 2
\SetRowColor{LightBackground}
W{\"a}hlt das Wort an der akt. Coursorposition aus. & STRG + d \tn 
% Row Count 7 (+ 3)
% Row 3
\SetRowColor{white}
Mehrfaches drücken von STRG + d & Fügt jeweils die n{\"a}chste Wiederholung des akt. Wortes ein \tn 
% Row Count 10 (+ 3)
% Row 4
\SetRowColor{LightBackground}
N{\"a}chste Wiederholung nicht ausw{\"a}hlen & STRG + u \tn 
% Row Count 12 (+ 2)
% Row 5
\SetRowColor{white}
W{\"a}hlt alles zwischen Klammern aus & STRG + SHIFT + m \tn 
% Row Count 14 (+ 2)
% Row 6
\SetRowColor{LightBackground}
Akt. Auswahl mit \textless{}p\textgreater{} Tag umhüllen & STRG + SHIFT + w \tn 
% Row Count 16 (+ 2)
\hhline{>{\arrayrulecolor{DarkBackground}}--}
\end{tabularx}
\par\addvspace{1.3em}

\begin{tabularx}{8.4cm}{x{3.76 cm} x{4.24 cm} }
\SetRowColor{DarkBackground}
\mymulticolumn{2}{x{8.4cm}}{\bf\textcolor{white}{Suchen und Ersetzen}}  \tn
% Row 0
\SetRowColor{LightBackground}
Suchpanel {\"o}ffnen & STRG + f \tn 
% Row Count 1 (+ 1)
% Row 1
\SetRowColor{white}
Suche in allen Dateien {\"o}ffnen & STRG + SHIFT + f \tn 
% Row Count 3 (+ 2)
% Row 2
\SetRowColor{LightBackground}
N{\"a}chstes Ergebnis ausw{\"a}hlen & F3 \tn 
% Row Count 5 (+ 2)
% Row 3
\SetRowColor{white}
Vorhergehendes Ergebnis ausw{\"a}hlen & SHIFT + F3 \tn 
% Row Count 7 (+ 2)
% Row 4
\SetRowColor{LightBackground}
Suchen und Ersetzen Panel {\"o}ffnen & STRG + h \tn 
% Row Count 9 (+ 2)
% Row 5
\SetRowColor{white}
Ausgew{\"a}hltes ersetzen & ALT + F3 \tn 
% Row Count 11 (+ 2)
% Row 6
\SetRowColor{LightBackground}
Inkrementielle Suche & STRG + i \tn 
% Row Count 13 (+ 2)
% Row 7
\SetRowColor{white}
Auswahl nutzen um Feld zu finden & STRG + e (nur engl. Tastaturen) \tn 
% Row Count 15 (+ 2)
% Row 8
\SetRowColor{LightBackground}
Auswahl nutzen um Feld zu ersetzen & STRG + SHIFT + e (nur engl. Tastaturen) \tn 
% Row Count 17 (+ 2)
\hhline{>{\arrayrulecolor{DarkBackground}}--}
\end{tabularx}
\par\addvspace{1.3em}

\begin{tabularx}{8.4cm}{x{4 cm} x{4 cm} }
\SetRowColor{DarkBackground}
\mymulticolumn{2}{x{8.4cm}}{\bf\textcolor{white}{Editing}}  \tn
% Row 0
\SetRowColor{LightBackground}
Zeile ausschneiden & {\bf{Cursor an den Anfang der Zeile setzten}} STRG + x \tn 
% Row Count 3 (+ 3)
% Row 1
\SetRowColor{white}
Akt. Zeile ausschneiden & STRG + SHIFT + k \tn 
% Row Count 5 (+ 2)
% Row 2
\SetRowColor{LightBackground}
In akt. Zeile alles vor dem Cursor ausschneiden & STRG + k \textgreater{} BACKSPACE \tn 
% Row Count 8 (+ 3)
% Row 3
\SetRowColor{white}
In akt. Zeile alles nach dem Cursor ausschneiden & STRG + k \textgreater{} k \tn 
% Row Count 11 (+ 3)
% Row 4
\SetRowColor{LightBackground}
Blockkommentar & {\bf{Inhalt ausw{\"a}hlen und}} STRG + / (nur engl. Tastaturen) \tn 
% Row Count 14 (+ 3)
% Row 5
\SetRowColor{white}
Kommentar & {\bf{Cursor an den Anfang der Zeile setzten und}} STRG + / (nur engl. Tastaturen) \tn 
% Row Count 18 (+ 4)
% Row 6
\SetRowColor{LightBackground}
Zeile duplizieren & STRG + SHIFT + d \tn 
% Row Count 19 (+ 1)
% Row 7
\SetRowColor{white}
Nachfolgende Zeile an akt. Zeile anh{\"a}ngen & STRG + j \tn 
% Row Count 22 (+ 3)
\hhline{>{\arrayrulecolor{DarkBackground}}--}
\end{tabularx}
\par\addvspace{1.3em}

\begin{tabularx}{8.4cm}{X}
\SetRowColor{DarkBackground}
\mymulticolumn{1}{x{8.4cm}}{\bf\textcolor{white}{US Keyboard Layout}}  \tn
\SetRowColor{LightBackground}
\mymulticolumn{1}{p{8.4cm}}{\vspace{1px}\centerline{%\includegraphics[width=5.1cm]{/web/www.cheatography.com/public/uploads/the-exilant_1455809934_teclatUKok.png}
}} \tn 
\hhline{>{\arrayrulecolor{DarkBackground}}-}
\end{tabularx}
\par\addvspace{1.3em}


% That's all folks
\end{multicols*}

\end{document}
