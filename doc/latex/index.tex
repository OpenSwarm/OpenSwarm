\hypertarget{index_intro_sec}{}\subsection{Introduction}\label{index_intro_sec}
Open\+Swarm is an easy-\/to-\/use event-\/driven preemptive operating system for miniature robots. It offers abstract hardware-\/independent functions to make user code more extendible, maintainable, and portable. The hybrid kernel provides preemptive and cooperative scheduling, asynchronous and synchronous programming models with events, and inter-\/process communication functions. ~\newline
~\newline
 Open\+Swarm was created during the Ph\+D of Stefan M Trenkwalder (\href{http://trenkwalder.tech}{\tt http\+://trenkwalder.\+tech}) at the University of Sheffield (\href{http://www.sheffield.ac.uk/}{\tt http\+://www.\+sheffield.\+ac.\+uk/}) under the Supervision of Dr. Roderich Gross and Dr. Andreas Kolling.

The code of Open\+Swarm can be basically divided into 3 different modules\+:
\begin{DoxyItemize}
\item \hyperlink{group__process}{Process Management}
\item \hyperlink{group__events}{Event Management}
\item \hyperlink{group__io}{I/\+O Management} (This includes device specific sensors and actuators)
\end{DoxyItemize}

All modules are, then, combined in Open\+Swarm\textquotesingle{}s \hyperlink{group__base}{Kernel} .

\hypertarget{index_link_sec}{}\subsection{Links}\label{index_link_sec}

\begin{DoxyItemize}
\item \href{http://www.openswarm.org/}{\tt http\+://www.\+openswarm.\+org/} The official Open\+Swarm website
\item \href{http://trenkwalder.tech/}{\tt http\+://trenkwalder.\+tech/} The academic webpage of Stefan Trenkwalder
\item \href{http://naturalrobotics.group.shef.ac.uk/}{\tt http\+://naturalrobotics.\+group.\+shef.\+ac.\+uk/} The website of the research group
\item \href{http://openswarm.org/license/}{\tt http\+://openswarm.\+org/license/} The link to the newest license (in case it changed)
\end{DoxyItemize}\hypertarget{index_base_license}{}\subsection{License}\label{index_base_license}
L\+I\+C\+E\+N\+S\+E\+: adapted Free\+B\+S\+D License (see \href{http://openswarm.org/license}{\tt http\+://openswarm.\+org/license})~\newline
Copyright (c) 2015, Stefan M. Trenkwalder~\newline
All rights reserved.\hypertarget{index_base_thanks}{}\subsection{Thanks}\label{index_base_thanks}
Open\+Swarm is part of the Ph\+D of Stefan M. Trenkwalder (\href{http://trenkwalder.tech}{\tt http\+://trenkwalder.\+tech}) who is recipient of a D\+O\+C Fellowship of the Austrian Academy of Sciences (\href{http://www.oeaw.ac.at/}{\tt http\+://www.\+oeaw.\+ac.\+at/}). 